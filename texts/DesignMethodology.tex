The methodology employed in this study consists of three phases composed of several activities.
In the first phase, the ontology design and development are carried out following a set of steps.
In the second phase, ontology validation is accomplished by an iterative refining process.
In the third phase, the ontology is implemented in a software application using an API for easier integration.

\subsection{Phase I - Design and Development} \label{phasesChapter}

The first phase is carried out using an ontology-based platform such as Prot�g� and OWL for the ontology representation.

The first step consists in the definition of the ontology scope by identifying the domain of the ontology, its purpose and functionalities and the end-users.

In the second step should be considered the possibility of reusing existing ontologies instead of developing one from scratch.
Many ontologies may already be available and can be easily imported into an ontology-based platform.

In the third step is performed the identification of the key concepts of the ontology being developed.
This task can be done following a top-down, bottom-up or middle-out approach in the elicitation of the different concepts.

In the forth step are identified concept to properties relationships.
This is carried out by defining and developing properties that represent relationships between one or more concepts.

In the fifth step are identified concept to instance relationships.
This step is similar to the last, but moves one step further into the lower-level, more detailed and complex relationships, since an instance is a specific realisation of a concept.

\subsection{Phase II - Validation}
 
The second phase encompasses various validation tasks that should be executed in a systematic approach.
Employment of metrics for ontology validation.
The usage of reasoners for correct verification of consistency of the developed ontology or ontologies.
Assessment of functional and non-functional requirements.

\subsection{Phase III - Implementation}

The last phase consists in the implementation of the developed ontology.
A suitable ontology API should be selected according to the appropriate application being developed that will ease the integration process of both artifacts.


